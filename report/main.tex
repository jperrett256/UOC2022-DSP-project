\documentclass{article}
\usepackage[utf8]{inputenc}
\usepackage{a4wide,parskip,underscore,xspace}
\usepackage{hyperref}
\hypersetup{
    colorlinks=true,
    linkcolor=blue,
    urlcolor=blue,
    citecolor=black
}

\newcommand*{\code}[1]{\texttt{#1}}
\newcommand*{\TODO}{\textbf{TODO}\xspace}

\title{Digital Signal Processing - Assignment 4a}
\author{Joshua Perrett (jdp63)}
\date{January 2023}

\begin{document}

\maketitle

\section*{Introduction}

For the mini-project, I opted for the first option: ``HDMI video cable eavesdropping''. The steps provided are fairly clear, and so I will describe my results more or less in the order that corresponds to the instructions given.

\section*{Manual Adjustment of Parameters}

\subsection*{Resampling}

I initially tried performing perfect sinc interpolation, which worked fine for a test input, but took far too long on the IQ data (even for just a couple of frames).

% TODO show perfect sinc interpolation code

After looking for more computationally-feasible approaches to resampling, I found a resource that implied that changing ``the sampling rate by a rational factor'' required interpolating and then decimating \cite{dspguru-resampling}. If the rational factor has a very large numerator, this process can require a great deal of memory. To mitigate this, the process can be divided into multiple stages, by splitting the factor into a series of smaller factors with smaller numerators and denominators \cite{dspguru-resampling}.

\textbf{TODO} mention:
\begin{enumerate}
\item you tried
\item cite the source that mentioned you could potentially split the resampling into (potentially multiple) upsampling and downsampling stages
\item mention that the DSP.jl library appears to do just this
\item \textbf{TODO} if you get around to it, show and compare different ways of resampling (linear, sinc, kaiser)
\end{enumerate}

\subsection*{Parameter Values}

Adjusting by eye, I found that $f_p$ was +0.1\% over the expected 25.175 MHz. By vertically joining two separate frames that were several frames apart, I updated my estimate to +0.997\%.

Adjusting by \TODO discuss alignment (achieved by discarding samples).

\textbf{TODO} show code, show figure

\subsection*{Result}

\section*{Automation Adjustment of Parameters}

Mention autocorrelation, how that has worked for you, why it worked.
Talk about how you found the range of values for $f_s / f_h$ and $f_s / f_v$.

\textbf{TODO} Autocorrelation of complex values? Cross correlation between frames?

\section*{TODO}

\section*{Conclusion}

\newpage
\appendix
% TODO appendices if needed

\newpage
\bibliographystyle{plain}
\bibliography{refs}

\end{document}
